%%%%%%%%%%%%%%%%%%%%%%%%%%%%%%%%%%%%%%%%%%%%%%%%%%%%%%%%%%%%%%%%%
%  _____       ______   ____									%
% |_   _|     |  ____|/ ____|  Institute of Embedded Systems	%
%   | |  _ __ | |__  | (___    Wireless Group					%
%   | | | '_ \|  __|  \___ \   Zuercher Hochschule Winterthur	%
%  _| |_| | | | |____ ____) |  (University of Applied Sciences)	%
% |_____|_| |_|______|_____/   8401 Winterthur, Switzerland		%
%																%
%%%%%%%%%%%%%%%%%%%%%%%%%%%%%%%%%%%%%%%%%%%%%%%%%%%%%%%%%%%%%%%%%


% The ZHAW INES latex style scrip
\documentclass{clsFile/zhawines}

% Command redefinitions

% To use new glossaries package uncomment this line. This requires Perl to be installed on the computer (http://strawberryperl.com/)
% Comment this line if you want to use the old manual glossary (see file /content/glossar)
%\newcommand*{\USEGLOSSARIES}{} 

%Include your own packages that are needed for you document, here the packages for the example are added
\usepackage{scrlayer-scrpage}
\usepackage{listings}
\usepackage{amsmath}
\usepackage[squaren]{SIunits}
\usepackage{graphicx}
\usepackage{array}
\usepackage{float}
\usepackage{pbox}
\usepackage{caption}
\usepackage{subcaption}
\ifdefined\USEGLOSSARIES
	\usepackage[acronyms]{glossaries}
	\makeglossaries
	%%%%%%%%%%%%%%%%%%%%%%%%%%%%%%%%%%%%%%%%%%%%%%%%%%%%%%%%%%%%%%%%%
%  _____       ______   ____									%
% |_   _|     |  ____|/ ____|  Institute of Embedded Systems	%
%   | |  _ __ | |__  | (___    Wireless Group					%
%   | | | '_ \|  __|  \___ \   Zuercher Hochschule Winterthur	%
%  _| |_| | | | |____ ____) |  (University of Applied Sciences)	%
% |_____|_| |_|______|_____/   8401 Winterthur, Switzerland		%
%																%
%%%%%%%%%%%%%%%%%%%%%%%%%%%%%%%%%%%%%%%%%%%%%%%%%%%%%%%%%%%%%%%%%

% Glossary Eintraege
\newglossaryentry{akronym}{name=akronym,plural=akronyme, description={Ein Akronym ist ein aus den Anfangsbuchstaben mehrerer Wörter gebildetes Kurzwort}}


% Akronym Eintraege
\newacronym{zhaw}{ZHAW}{Zürcher Hochschule für Angewandte Wissenschaften}
\newacronym{ines}{InES}{Institute of Embedded Systems}
\newacronym{tofcamera}{TOF Camera}{Time of Flight Camera}
\newacronym{fpga}{FPGA}{Field Programmable Gate Array}
\newacronym{pcie}{PCIe}{PCI Express}
\newacronym{pci}{PCI}{Peripheral Component Interconnect}
\newacronym{usb}{USB}{Universal Serial Bus}
\newacronym{CSI}{CSI}{Camera Serial Interface}


\fi

\DeclareCaptionFormat{myformat}{#1#2#3}
\captionsetup[figure]{format=myformat}

\begin{document}
%How the hyperlinks are shown, remove this or change it if you want the colorful links
\hypersetup{
    colorlinks = true,
    urlbordercolor = {1 1 1},
    linkbordercolor = {white},
}

%Select language, all generated content will be in the selected language
\selectlanguage{english} %possible german/english

% Generate your title page, it takes the arguments author, title
% and email. The title page is then generated with maketitle
\author{Marcel Wegmann \newline }
\title{Addition to: Augmented Reality Platform using Sensor Fusion and embedded GPU Processing}
\email{wegr@zhaw.ch}
\newcommand{\version}{1.0}

% Add Copyright informations and Contact infos
 \newcommand{\confidential}{no}  %write yes to have a confidential mark on every normal page

\maketitle


% Generate the header and footer of the document
\headings

% Stop page numbering for abstract and anti-plagiarism statement
\pagenumbering{gobble}



%Include table of content
\tableofcontents
\newpage
% Start page numbering
\pagenumbering{arabic}
% Starts the page count from here, pdf viewers will be able to jump to the corret page.
\setcounter{page}{1}
\setcounter{MaxMatrixCols}{17}
%Include all chapters. Chapters itself are in the content subfolder
% %%%%%%%%%%%%%%%%%%%%%%%%%%%%%%%%%%%%%%%%%%%%%%%%%%%%%%%%%%%%%%%%%
%  _____       ______   ____									%
% |_   _|     |  ____|/ ____|  Institute of Embedded Systems	%
%   | |  _ __ | |__  | (___    Wireless Group					%
%   | | | '_ \|  __|  \___ \   Zuercher Hochschule Winterthur	%
%  _| |_| | | | |____ ____) |  (University of Applied Sciences)	%
% |_____|_| |_|______|_____/   8401 Winterthur, Switzerland		%
%																%
%%%%%%%%%%%%%%%%%%%%%%%%%%%%%%%%%%%%%%%%%%%%%%%%%%%%%%%%%%%%%%%%%

\chapter{LaTeX Kurzanleitung}\label{chap.anleitung}
Dieses Kapitel führt mit Beispielcode in den LaTeX Code ein, und kann während der Erstellung des Dokuments gelöscht werden.\footnote{Verbesserungsvorschläge bitte an hegt@zhaw.ch senden}

Die nachfolgende Berichtstruktur wurde aus der Vorlage\footnote{Berichtstruktur Vorlage, Stand: August 2011} der \href{https://intra.zhaw.ch/departemente/school-of-engineering/studium-standort-winterthur/studierende/projektarbeit-bachelorarbeit.html}{PA/BA Termin-Webseite} vom ZHAW Intranet entnommen.

(): alle in Klammer aufgeführten Einträge sind situativ anzupassen



Das ist ein kleiner Text um zu zeigen, wie die Enter eingebracht werden.\\
Ich finde Latex schwierig.\\
Der Start ist echt eine Herausforderung. \\
Mensch ist das Komplex, echt was für Profis. % Hier brauchts ja kein Enter mehr, sonnst heisst es Overfull \hbox



\section{Visio Vektorgraphik einfügen}\label{visio}
(Graphik auswählen) Speichern unter -> PDF -> Optionen.. -> Auswahl\\
Mit Adobe Akrobat öffnen: Erweitert -> Druckproduktion -> Seiten beschneiden -> Weisse Ränder entfernen -> OK -> Ctrl-S

\begin{figure}[H]
	\centering
		\includegraphics[width=0.8\textwidth]{images/visio/visio.pdf}
	\caption{Ideenskizze}
	\label{fig.Skizze}
\end{figure}

So kann die Abbildung~\ref{fig.Skizze} referenziert werden. Bei der PDF Erstellung ist darauf zu achten, dass LaTeX nur Versionen bis 1.4 voll unterstützt. 



\subsection{Graphiken in LaTeX zuschneiden}\label{zuschneiden}
Mit dem Befehl Clip kann eine Graphik auch in LaTeX zugeschnitten werden:

\begin{figure}[H]
	\centering
		\includegraphics[width=0.9\textwidth, clip=true, trim = 80 10 0 10]{images/visio/visio.pdf}  % trim lm bm rm tm (left, bottom, right, top)
	\caption{clip=true, trim = 60 10 0 10}
	\label{fig.SkizzeZugeschnitten}
\end{figure}



\subsection{Mehrere Bilder nebeneinander}\label{nebeneinander}
Dank Minipages können mehrere Bilder auch nebeneinander sein:

%Zwei Bilder nebeneinander http://latex.mschroeder.net/
\begin{figure}[H]
  \centering
  \begin{minipage}[b]{0.45\textwidth}
    \includegraphics[scale=0.15]{images/photoshop/Skizze.jpg}
    \caption{Visir10b Detector}
    \label{Visir10bDetector} 
  \end{minipage} % Hier darf keine Leerzeile zwischen den beiden Minipages liegen!
  \begin{minipage}[b]{0.45\textwidth}
    \includegraphics[scale=0.15]{images/photoshop/Skizze.jpg} 
    \caption{Visir10b Model}
    \label{Visir10bModel} 
  \end{minipage}
  \caption{Visir 10 mit optimiertem Reflektor}
  \label{fig.Visir10b}
\end{figure}


Literaturverweis: \citep{analog_devices_dac} \citep{microchip_spi}

\section{Tabellen aufbauen}\label{tabelle}
Kleine Tabelle:

\begin{table}[ht] \centering
	\begin{tabular}{|p{3cm}|p{.5cm}|p{.5cm}|p{.5cm}|p{.5cm}|p{.5cm}|p{.5cm}|p{.5cm}|p{.5cm}|p{.5cm}|p{.5cm}|} \hline
		\rowcolor{gray} Modul & M01 & M02 & M03 & M04 & M05 & M06 & M07 & M08 & M09 & M10 \\
		\hline
		FPGA\_DATEN & & & & & X & X & & X & X & \\
		\hline
		IRQ & X & X & X & & X & & & & X & X \\
		\hline
		Nachbar Core & & & & X & & X & & X & & \\
		\hline
	\end{tabular}
	\caption{Port Schwierigkeiten der Funkmodule}
	\label{tab:portprobleme}
 \end{table}	

Die nachfolgende longtable kann sich über mehrere Seiten erstrecken.

\begin{longtable}{|p{1.1cm}|p{4cm}|p{4cm}|p{4cm}|} 
					\hline
					\rowcolor{gray} Typ & Variante A & Variante B & Variante C
					\\ \hline
					& \textbf{Vorteile:} 
							\begin{itemize}
								\item[+] hohe Spannungen
							\end{itemize}							
							\textbf{Nachteile:}
							\begin{itemize}
								\item[-] Grosse Abmessung
							\end{itemize}
					& 	\textbf{Vorteile:} 
							\begin{itemize}
								\item[+] einfache Montage
							\end{itemize}							
							\textbf{Nachteile:}
							\begin{itemize}
								\item[-] max. 2A Eingangsstrom
							\end{itemize}
					&	\textbf{Vorteile:} 
							\begin{itemize}
								\item[+] hoher Strom
							\end{itemize}							
							\textbf{Nachteile:}
							\begin{itemize}
								\item[-] max. 12 V Eingangsspannung
							\end{itemize}\\ \hline
						Zeit & 2 h & 5 h & 3 h \\ \hline
						Preis	& 520 CHF/Stück & 800 CHF/Stück &	360 CHF/Stück\\ \hline
				\caption{Morphologischer Kasten für die Speisung}
				\label{tab:morphkasten}
			\end{longtable}	

Diese Art von Tabelle erstreckt sich immer auf der ganzen Seitenlänge:

\begin{tabularx}{\textwidth}{XXl}
  Salat & Schnecke & Igel\\
  Montag & Hier ist ein langes Wort & Dienstag
\end{tabularx}



\section{Code Listings aufbauen}\label{listing}

\begin{lstlisting}[
    language=C++,
    caption={Test Kommandozeilen Ausgabe},
    label={list.Testoutput}
]
/********************************************************/
/* Name        : M07Setup                               */
/* Description : EM9201 init for adress and pck         */
/* Input       : targetadr (DevAdr_M00 - DevAdr_M39)    */
/*               drate (Drate_M00 - Drate_M39)          */
/* Ouput       : -0x01 -> Setup OK                      */
/*               -0x5E -> Setup Error Channel write     */
/*               -0x6E -> Setup Error power write       */
/*               -0x7E -> Error in Device Address       */
/*               -0x8E -> Error in Peer Address         */
/********************************************************/
\end{lstlisting}

Formula $e = \sqrt{a{^2} - b^{2}}$


Diese Textstelle ist sehr interessant.\label{interessant}\\
Hier wird auf die Textstelle~\ref{interessant} verwiesen, die sich auf der Seite~\pageref{interessant} befindet.




\section{Citation nach IEEE}\label{citation}

Das ist ein \cite{analog_devices_dac} Verweis aufs Literaturverzeichnis. Ein anderes Beispiel ist das hier \cite{mirrorcle_userguide}.





Das ist eine Aufzählung:
\begin{itemize} %
	\item Erste Zeile
	\item Zweite Zeile
	\item Dritte Zeile
\end{itemize}


\begin{enumerate}
\item erstens
\item zweitens
\end{enumerate}

%https://parma.zhaw.ch/svn/zhw_latex/
%gmanstyle nötig unter start od begin einbinden


Das ist eine verschachtelte Aufzählung:
\begin{description}
		\item [Register Performance] Alle Signale die das FPGA nicht verlassen, also von FF zu FF weitergeleitet werden. Daraus ergibt sich die maximale Taktfrequenz F\textsubscript{MAX}.

		\item [Externes Timing] FPGA Ein- und Ausgänge
		\begin{itemize}
			\item Ausgänge = Von FF's durch Logik zu Ausgängen (t\textsubscript{CO})
			\item Eingänge = Von Eingängen durch Logik zu FF's (t\textsubscript{SU}, t\textsubscript{H})
			\item Durchgänge = kombinatorische Pfade durch das FPGA (t\textsubscript{PD})
		\end{itemize}
\end{description}

\ifdefined\USEGLOSSARIES

	\section{Akronyme und Glossar}\label{acronymsandglossaries}
	
	\subsection{Akronyme}\label{acronymsandglossary.acronyms}
	Ein Akronym ist ein aus den Anfangsbuchstaben mehrerer Wörter gebildetes Kurzwort. 
	
	
	Einträge der Akronyme können im Text folgendermassen dargestellt werden:
	\begin{itemize}
		\item Das \acrshort{ines} ist Teil der \acrshort{zhaw}.
		\item Das \acrlong{ines} ist Teil der \acrlong{zhaw}.
		\item Das \acrfull{ines} ist Teil der \acrfull{zhaw}.
	\end{itemize}
	
	\subsection{Glossar}\label{acronymsandglossary.glossaries}
	Das Glossar ist eine Liste erklärungsbedürftiger und für die Arbeit
	relevanter Begriffe zusammen mit den zugehörigen Erklärungen oder
	Übersetzungen.
	
	
	Einträge des Glossars können im Text folgendermassen dargestellt werden:
	\begin{itemize}
		\item \textbf{\gls{akronym}}: Singular mit Kleinbuchstaben
		\item \textbf{\Gls{akronym}}: Singular, erster Buchstabe gross
		\item \textbf{\glspl{akronym}}: Plural mit Kleinbuchstaben
		\item \textbf{\Glspl{akronym}}: Plural, erster Buchstabe gross
	\end{itemize}

\fi


 % Remove for final document
%!TEX root = ../additional_work.tex
\chapter{Introduction}
\label{sec:Introduction}
After implementing the entire pipeline during the Master's thesis, the system is targeted to become a suitable demo-object for the InES HPMM group. As the position estimation has not been stable, additional functions need to get implemented.
\begin{itemize} %
	\item Performance improvement (currently, only 10FPS of ToF-Camera-Data can be processed, ToF-Camera itself delivers up to 20fps)
	\item Use of ToF Data to estimate position with auto-generating a database of noteable feature clusters to generate a position estimation for stabilizing the translation-data in the Kalman filter
	\item Maybe posh it up visually
\end{itemize}

\chapter{Performance}
\label{sec:Terminology}
The Texture dataflow for the Main-Camera and the Projection objects have been rewritten. Initially, the texture data was written to host-memory space and copied by Vulkan again to the device-memory space. Vulkan-Cuda interoperability allows a zero-copy mode between the two API.\\
A multitude of functions is required for doing so, listed in calling order. 

\begin{itemize} %
	\item At Vulkan VkInstance creation: get FunctionDescriptor from VkInstance for: "vkGetPhysicalDeviceProperties2" (Used to query DeviceUUID for comparison with CUDA DeviceUUID -> Both need to be on the same device)
	\item At Vulkan VkInstance creation: get FunctionDescriptor from VkInstance for: "vkGetMemoryFdKHR" (Used to get Vulkan-Memory File Descriptor. Buffer managed by Vulkan, CUDA gets access to it)
	\item After Vulkan VkLogicalDevice creation: get FunctionDescriptor from VkLogicalDevice for: "vkGetSemaphoreFdKHR" (Semaphores are managed by vkLogicalDevice, multiple logical devices can be invoked by one instance)
    \item TextureImage object's staging-buffer created with ExternalMemory Bits ("VK\_SHARING\_ MODE\_EXCLUSIVE" and "VK\_EXTERNAL\_MEMORY\_HANDLE\_TYPE\_OPAQUE\_\\FD\_BIT") for both the MainCamera and the ProjectedImage objects.
    \item SetCudaVkDevice will then validate that everything will run on the same device (UUID- Comparison + Flag-Check if everything is supported. Pointless on a Jetson Device, but necessary on a Desktop)
    \item cudaVkImportMem will then import the external memory and incorporate it as a Cuda Memory with size and everything needed.
\end{itemize}

Even though, the former solution incorporated a shared buffer between Host and Cuda-Device, eliminating the copy from host to Vulkan-Device greatly improved performance. The Vulkan-Device memory is now shared with Cuda. \\
Synchronization is currently solved by forcing cuda to finish its calculations by a "cudaDeviceSynchronize()" call, before calling the Vulkan Function that copies the data from the Vulkan-Device Staging Buffer to the VkImage object. Copying from CUDA directly to the VkImage object may be possible but resulted in segmentation faults. Semaphores during the render stage might be required. The performance with only the staging-buffer being shared is sufficient - roughly 60fps could be achieved, the ToF Camera only delivers up to 20fps - no more investigation is done in that regard.

\chapter{Clustering}
For the algorighm of the Master's Thesis (velocity and rotation-velocity estimation), Sift features already get extracted from the ToF image, and mapped into a 3D space. Rejecting any feature metric other than it's spatial location should allow detecting entire objects (Lamps, Chairs etc) in the scene. Assuming that these objects do not move, it should be possible to estimate the current position and orientation of the camera. As rotation estimation already works fine, the focus lies on position estimation.
\section{Pruned K-Means Clustering}
The algorithm of K-Means clustering usually expects the user to know the number of feature-clusters apriori. Smart placement of seed points - one for each cluster - in the feature space allows the algorithm to find the closest points to these seeds. Each point gets allocated to a cluster. Filling the feature space with a vast amount of clusters allows filtering (or pruning) out small clusters aposteriori. The amount and positioning of the seed points influences whether a larger object gets recognized as one or multiple clusters. 

\section{Storing prominent cluster centers}




% %!TEX root = ../doc.tex
\chapter{Motivation}

The AR game Pokemon Go became a massive success in 2016 and remained the most famous AR game to date. Its success even brought investors to buy the stock without prior research. Nintendo's market value almost doubled before a warning given by Nintendo themselves that their economic stake in Pokemon Go is limited.\cite{Nintendo_stock}\\ 
Another example is the infamous Google Glass, a specialized AR goggle used to browse the web or take photographs.\cite{Google_Glass} The unwillingness of some users of Google Glass to take off the goggles during a conversation or secretly filming people in public sparked criticism to the extent that the word "Glasshole" emerged.\cite{Glasshole} Google themselves reacted by writing guidelines for Google Glass’ early adopters.\cite{Glasshole}\\
Microsoft opted for a bulkier design for the HoloLens\cite{Hololens} and targeted it towards a professional environment, for example, to support Airbus technicians at maintenance\cite{AirbusHololens}.\\
Apple did not unveil any AR goggles yet but demonstrates their advancements in augmented reality on their LiDaR equipped iPhones and iPads.\cite{AppleLidar}\\
While these companies provide specialized hardware and programming interfaces, the mechanisms behind them are corporate secrets. Leveraging the power of off-the-shelf compute modules and sensors allows creating a minimal working prototype for augmented reality.




% Generate title for lists
\cleardoublepage
\phantomsection
\addcontentsline{toc}{chapter}{Verzeichnisse}


\end{document}
