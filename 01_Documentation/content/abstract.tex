%!TEX root = ../doc.tex

\chapter*{Abstract}
\label{sec:Abstract}
Augmented Reality is the concept of enhancing the real world with virtual objects or information with projections into a viewfinder or through specialized goggles. Simpler forms of Augmented Reality – like a heads-up display in a car – do not need to estimate the camera's motion, an object, or the user.  However, more elaborate implementations of Augmented Reality need to track things and, more importantly, the camera's movement itself. The applications in which Augmented Reality could be leveraged range from social interaction over pedestrian navigation to various use cases in different professions. Multiple companies already have shown closed source or custom-tailored programming interfaces, either running on smartphones or shipped with industry-targeted goggles. The tracking of real-world objects surfaces or is possible with the provided interfaces, but the algorithms behind the different functions are corporate secrets.\\ 
This thesis describes an approach for an end-to-end pipeline in a prototype of an Augmented Reality platform without using commercial interfaces. A time-of-flight camera provides a depth image that allows reconstruction of the recorded scene as a cloud of SIFT features. Frame-by-frame analysis of the point cloud estimates the camera's motion by highly parallel processing and a three-dimensional extension of the RANSAC algorithm. An accelerometer and a gyroscope provide additional data, fused with a Kalman filter to improve the motion estimation. A regular color camera acts as a viewfinder, and Vulkan renders the result to a monitor.\\
Enhancing the matching quality of SIFT features between consecutive frames of a time-of-flight camera using a three-dimensional RANSAC algorithm led to over two times as many correct matches. Despite noisy camera data, the estimation of the camera rotation and translation of the time-of-flight camera based on these matches works as demonstrated in the thesis. The sensor fusion with the Kalman filter works as intended for rotations. Still, it fails for translations because of the system's low sampling rate and the accelerometer's hysteresis, failing to compensate appropriately for gravity.

\chapter*{Preface}
\label{sec:Vorwort}
I am thankful to my supervisor, Prof. Dr. Matthias Rosenthal, for allowing this deep dive into Augmented Reality and the support given during this thesis. Furthermore, I am grateful for the support and advice from the ZHAW InES HPMM team.\\Special thanks, especially to Lukas Neuner, for his valuable inputs during this thesis and for proofreading this document. Additional thanks are given to Alexey Gromov for his support in setting up the Jetson Xavier and proofreading this document. I am thankful for the chance of gaining further experience in CUDA, Vulkan and for the time given to learn new topics in the math involved in three-dimensional rendering. \\
In addition, I thank all my friends and family members for giving support and motivation.