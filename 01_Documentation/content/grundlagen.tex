%!TEX root = ../doc.tex
\chapter{Fundamentals}
\label{sec:Fundamentals}
The following chapter describes methods and technologies that are used within this thesis.
\section{Mathematics for Rotation and Translation}
\label{sec:LinAlgRotation}
Augmented reality relies on having accurate positional and angular information, to estimate the requred size and warp of a virtual object that is projected into the real world. To assist the positional tracking, a MEMS-module containing a gyroscope and an accelerometer provides rotation and acceleration information to the system. 
\subsection{Euler Rotations and Linear Algebra}
A common way to calculate rotations and translations are matrix-vector multiplications. The standard matrices for rotating with the angle $\phi$ around $X$, $Y$ and $Z$ are shown in the following:
\begin{equation*}
    A_{rot,X} = 
    \begin{bmatrix}
    1 & 0 & 0 \\
    0 & cos \phi & -sin \phi \\
    0 & sin \phi & cos \phi
    \end{bmatrix}
    \quad
    A_{rot,Y} = 
    \begin{bmatrix}
    cos \phi & 0 & sin \phi \\
    0 & 1 & 0 \\
    -sin \phi & 0 & cos \phi
    \end{bmatrix}    
    \quad
    A_{rot,Z} = 
    \begin{bmatrix}
    cos \phi & sin \phi & 0 \\
    -sin \phi & cos \phi & 0 \\
    0 & 0 & 1
    \end{bmatrix} 
\end{equation*}
A combination of the three matrices leads to a rotation matrix with a rotation axis that is not strictly bound to $X$, $Y$ or $Z$. A matrix multiplication is not commutative, so the order of the multiplications matters. In the following example, the vector would be rotated first around $X$, then $Y$ and around $Z$ in the end - in the equation from right to left.
\begin{equation*}
    A_{rot} = 
    \begin{bmatrix}
    a_{0,0} & a_{0,1} & a_{0,2}  \\
    a_{1,0} & a_{1,1} & a_{1,2}  \\
    a_{2,0} & a_{2,1} & a_{2,2} 
    \end{bmatrix}
    =A_{rot,Z} \cdot A_{rot,Y} \cdot A_{rot,X}
\end{equation*}
With this matrix, a three dimensional vector can be rotated at once around an arbitrary axis for the desired angle. Applying this transformation to each vertex of a virtual 3D object results in a rotation of the whole object around the origin $(0,0,0)$.\\
\begin{equation*}
    \begin{pmatrix}
    x'  \\
    y'  \\
    z' 
    \end{pmatrix} 
    = 
    \begin{bmatrix}
a_{0,0} & a_{0,1} & a_{0,2}  \\
a_{1,0} & a_{1,1} & a_{1,2}  \\
a_{2,0} & a_{2,1} & a_{2,2} 
\end{bmatrix}
\cdot 
\begin{pmatrix}
    x  \\
    y  \\
    z 
    \end{pmatrix}
\end{equation*}
To avoid using an inhomogenous linear system for moving an object, a fourth dimension is needed. By extending the vectors with a $1$ and using the fourth column in the matrix to alter $X$, $Y$ and $Z$, these vector entries can be moved without applying any rotation. 
\begin{equation*}
    \begin{pmatrix}
        x+\Delta X  \\
        y+\Delta Y  \\
        z+\Delta Z  \\
        1
        \end{pmatrix} 
        = 
    \begin{pmatrix}
    x'  \\
    y'  \\
    z'  \\
    1
    \end{pmatrix} 
    = 
    \begin{bmatrix}
1 & 0 & 0 & \Delta X \\
0 & 1 & 0 & \Delta Y  \\
0 & 0 & 1 & \Delta Z  \\
0 & 0 & 0 & 1
\end{bmatrix}
\cdot 
\begin{pmatrix}
    x  \\
    y  \\
    z  \\
    1
    \end{pmatrix}
\end{equation*}
To combine the rotation matrix $A_{rot}$ with the translation matrix $A_{trans}$ gets placed top-left into the 4x4 unit matrix. Now, the rotation matrix also being a 4x4 matrix, rotations and translations can be chained up following the common laws of linear algebra.
\begin{equation*}
    \begin{pmatrix}
    x'  \\
    y'  \\
    z'  \\
    1
    \end{pmatrix} 
    = 
    \begin{bmatrix}
a_{0,0} & a_{0,1} & a_{0,2} & 0 \\
a_{1,0} & a_{1,1} & a_{1,2} & 0 \\
a_{2,0} & a_{2,1} & a_{2,2} & 0 \\
0 & 0 & 0 & 1
\end{bmatrix}
\cdot 
\begin{pmatrix}
    x  \\
    y  \\
    z  \\
    1
    \end{pmatrix}
\end{equation*}
The dependency on the order of the rotations poses a problem: The values returned by a gyroscope would need to be applied all at once and not one after another. 