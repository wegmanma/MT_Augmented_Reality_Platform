%!TEX root = ../doc.tex
\chapter{Motivation}
\label{sec:Motivation}
Multiple consulting companies like Deloitte and KPMG described virtual, augmented, and extended reality as possibly the biggest source of digital disruption since the smartphone\cite{KPMG_on_AR} and the next big thing of the digital environment\cite{Deloitte_on_AR}.\\
While augmented reality platforms already exist in consumer products, the know-how is developed within the walls of multi-billion tech companies like Apple or Microsoft. According to Bloomberg, Facebooks AR and VR, and hardware teams account for more than 6000 employees\cite{Bloomberg_on_AR}. \\
The AR game Pokemon Go became a massive success in 2016 and remained the most famous AR game to date. Its success even brought investors to buy the stock without prior research. Nintendo's market value almost doubled before a warning given by Nintendo themselves that their economic stake in Pokemon Go is limited.\cite{Nintendo_stock}\\ 
Another example is the infamous Google Glass, a specialized AR goggle used to browse the web or take photographs.\cite{Google_Glass} The unwillingness of some users of Google Glass to take off the goggles during a conversation or secretly filming people in public sparked criticism to the extent that the word "Glasshole" emerged.\cite{Glasshole} Google themselves reacted by writing guidelines for Google Glass’ early adopters.\cite{Glasshole}\\
Microsoft opted for a bulkier design for the HoloLens\cite{Hololens} and targeted it towards a professional environment, for example, to support Airbus technicians at maintenance\cite{AirbusHololens}.\\
Apple did not unveil any AR goggles yet but demonstrates their advancements in augmented reality on their LiDaR equipped iPhones and iPads.\cite{AppleLidar}\\
While these companies provide specialized hardware and programming interfaces, the mechanisms behind them are corporate secrets. Leveraging the power of off-the-shelf compute modules and sensors allows creating a minimal working prototype for augmented reality.

